% ------------------------------------------------------------
% Virlupus LaTeX Template - v1.0 (2025)
% A PDFLaTeX template for popular science, literature, and educational texts.
%
% This document is free software; you can redistribute it and/or modify
% it under the terms of the GNU General Public License as published by
% the Free Software Foundation; either version 2 of the License, or
% (at your option) any later version.
%
% Designed for elegant typesetting of A5 books with professional typography.
% Copyleft - Virlupus Software - 2025
% ------------------------------------------------------------

% Base packages and document settings
\documentclass[a5paper,12pt]{book}							% A5 paper and 12pt font improves readability

% We can change the margins, but it is not neccessary
\usepackage[a5paper,outer=2cm,inner=2.5cm,top=2.5cm,bottom=2.5cm]{geometry}

% \usepackage{romande}    									% For literary texts
% \usepackage{libertinus} 									% Universal: serif, math-friendly, elegant
\usepackage{mlmodern}     									% For scientific texts (stronger appearence)

\usepackage[T1]{fontenc}
\usepackage[utf8]{inputenc}
\usepackage[english]{babel}									% Select your language. I'm from the Czech Republic, sorry :-)
\usepackage[all]{nowidow}									% Typographic adjustment: widows and orphans don't look aesthetically pleasing
\usepackage{multicol}										% Sometimes we need multi-column formatting

\usepackage{longtable}										% For tables that span multiple pages
\usepackage{array}											% This package removes some limitations of the tabular environment

% Sometimes it's necessary to use math, so we prepare the environment here
\usepackage{amsfonts}
\usepackage{amsmath}
\usepackage{amssymb}

% An alphabetical index is the foundation of every scientific book
\usepackage{imakeidx}
\makeindex

% ------------------------------------------------------------
% Graphics for headers and footers
\usepackage{quotchap}										% Chapter quotation
\usepackage{fancyhdr}
\usepackage{pgfornament}
\pagestyle{fancy}

\renewcommand{\chaptermark}[1]{\markboth{#1}{}
}
\renewcommand{\sectionmark}[1]{\markright{#1}
}

\renewcommand\headrule{
	\hrulefill
	\color{black}
	\pgfornament[anchor=west, width=2.0cm]{87}
	\hrulefill
}

\fancyhf{}
\fancyhead[LE,RO]{\thesection}
\fancyhead[RE,LO]{\thepage}
\fancyhead[RO]{\textsl{\leftmark}}
\fancyhead[LE]{\textsl{\rightmark}}
\setlength{\headheight}{25.504pt}

\fancypagestyle{plain}

% ------------------------------------------------------------
% Hypertext links – table of contents, index, web links, etc.
\usepackage{graphicx}										% We need images
\usepackage[svgnames]{xcolor}								% Named colors

\usepackage{hyperref}
\usepackage[all]{hypcap}
\hypersetup{
	pdfborder=0 0 0,
	linkcolor=DarkSlateGrey,
	urlcolor=DarkSlateGrey,
	citecolor=DarkSlateGrey,
	colorlinks=true
}

% ------------------------------------------------------------
% Framed text for important math or other content
\usepackage{tcolorbox}
\tcbuselibrary{breakable}
\tcbuselibrary{theorems}
\tcbuselibrary{skins}
\tcbset{
	enhanced,
	parbox=false,
	left=0.5mm,
	right=0.5mm,
	boxrule=0.5mm,
	breakable,
	drop large lifted shadow,
	fonttitle=\bfseries\slshape,
	separator sign={\ $\bullet$}
}

% Framed environments for important content – you can also add icons
\newenvironment{grammar}[1]
{\begin{tcolorbox}[colframe=CadetBlue, colback=CadetBlue!5, title=Gramatika - #1]}
	{\end{tcolorbox}}

\newenvironment{remember}
{\begin{tcolorbox}[colframe=red, colback=AliceBlue]}
	{\end{tcolorbox}}

\newenvironment{note}
{\begin{tcolorbox}[colframe=CadetBlue, colback=AliceBlue]}
	{\end{tcolorbox}}

\newenvironment{important}
{\begin{tcolorbox}[colframe=Orange, colback=AliceBlue]}
	{\end{tcolorbox}}

\newtcbtheorem[number within=section]{theorem}{Theorem}%
{colback=Gray!5,colframe=CadetBlue,fonttitle=\bfseries\slshape}{th}

\newtcbtheorem[number within=section]{proof}{Proof}%
{colback=Gray!5,colframe=CadetBlue,fonttitle=\bfseries\slshape}{th}

\newtcbtheorem[number within=section]{define}{Definition}%
{colback=Gray!5,colframe=CadetBlue,fonttitle=\bfseries\slshape}{th}

% ------------------------------------------------------------
% Additional math operators for localization
\DeclareMathOperator{\tg}{tg}                       		% Tangent (in Central Europe)
\DeclareMathOperator{\arctg}{arctg}                 		% Inverse tangent
\DeclareMathOperator{\e}{e}                         		% Exponeniial function

\newcommand{\ux}[1]{\ensuremath{^{\mathrm{#1}}}}    		% Superscript with upright font
\newcommand{\lx}[1]{\ensuremath{_{\mathrm{#1}}}}    		% Subscript with upright font
\newcommand{\DD}[1]{\Delta #1}                      		% Difference (not derivative)
\newcommand{\DV}[2]{\frac{\Delta #1}{\Delta #2}}    		% Difference quotient

% ------------------------------------------------------------
% Packages for chemistry and physics units
\usepackage{chemexec}
\usepackage{physics}
\usepackage{physunits}

% ------------------------------------------------------------
% Graphics settings for your drawings
\usepackage{pgfplots}
\pgfplotsset{compat=1.15}
\usetikzlibrary{arrows}

% ------------------------------------------------------------
% Base path for your images
\graphicspath{{assets/}}

% ------------------------------------------------------------
% Here is the realy text of your book
\begin{document}
	
	% --------------------------------------------------------
	% Title page under your control
	
	\begin{titlepage}
		\begin{center}
			\vspace*{10mm}
			
			% Book title
			{\Huge The sample of a book}\\
			\vspace{5mm}
			
			% Try add any graphics or informations about
			\vfill
			
			% Author
			{\large \textbf{PhDr. Mgr. Virlupus Volchv}}\\
			\vfill
			
			% Some logo of an institute
			\includegraphics[width=0.1\textwidth]{"logo.png"}
			
			% Organization and data
			{\small Schola Versipellem\\}
			{\footnotesize Jihlava\\}
			{\scriptsize \today}\\
			{\scriptsize version 1.0}
		\end{center}
	\end{titlepage}
	
	% Contents list
	\tableofcontents
	
	% --------------------------------------------------------
	% The main document begins here
	
	% Quote for this chapter - width is 75% of content width
	\begin{savequote}[.75 \textwidth]
		Two things are infinite: the universe and the human stupidity.
		\qauthor{Albert Einstein}
	\end{savequote}
	
	% Chapter title
	\chapter{Shall we try physics?}
	
	What is physics? We can say that it is an attempt to understand the events that we see around us in the nature.
	
	\dots
	
	Let's look at force, it's just a mathematical construct, but it can facilitate a lot of calculations that are useful in practical applications. Whether it's space flights or children playing on a swing.
	
	\section{Newton's laws}
	
	Issac Newton left us several interesting insights that have been denied for centuries, mainly by church organizations.
	
	\index{Physics!Laws!Newton inertia}
	\index{Physics!Laws!Newton first law}
	\index{Newton inertia}
	\index{Newton!inertia}
	\index{Law of inertia}
	\index{Newton!first law}
	\begin{define}{Newton first law}{}
		An object at rest remains at rest, or if in motion, remains in motion at a constant velocity unless acted on by a net external force.
		
		Corpus omne perseverare in statu suo quiescendi vel movendi uniformiter in directum, nisi quatenus a~viribus impressis cogitur statum illum mutare.
	\end{define}
	
	\dots
	
	\index{Physics!Newton gravity}
	\index{Newton gravity}
	\index{Newton!gravity}
	\index{gravity}
	\begin{define}{Newton's gravity}{}
		Any particle of matter in the universe attracts any other with a force varying directly as the product of the masses and inversely as the square of the distance between them.
	\end{define}
	
	The equation for universal gravitation thus takes the form:
	
	\begin{remember}
		\[F = G\frac{m_1\cdot m_2}{r^2}\]
	\end{remember}
	
	where $F$ is the gravitational force acting between two objects, $m_1$ and $m_2$ are the masses of the objects, $r$ is the distance between the centers of their masses, and $G$ is the gravitational constant. 
	
	Over time, it was necessary to replace this theory with a more accurate one. Although Newton's equation is still valid, it is only a certain limit. The reason is that it does not respect the role of time and its results are sometimes inaccurate.
	
	\index{Physics!Einstein relativity}
	\index{Einstein!relativity}
	\index{general relativity}
	\index{relativity}
	\index{gravity}
	That's why we use Einstein's special and general relativity today.
	
	\begin{important}
		\[R_{\mu\nu} - \frac{1}{2}R\,g_{\mu\nu} + \Lambda\, g_{\mu\nu} = \frac{8\pi G}{c^4}\,T_{\mu\nu}\]
	\end{important}
	
	\dots
	
	\chapter{Let’s Try Mathematics}
	
	Geometry gives us powerful tools to describe space. One of the most famous results in geometry is the Pythagorean theorem.
	
	\index{Math!pythagorean theorem}
	\index{pythagorean theorem}
	\index{Theorems!pythagorean}
	\index{triangle}
	\begin{define}{Pythagorean Theorem}{}
		In a right-angled triangle, the square of the hypotenuse is equal to the sum of the squares of the other two sides.
		\[
		c^2 = a^2 + b^2
		\]
	\end{define}
	
	Let’s now look at a simple algebraic proof of this statement.
	
	\begin{proof}{Algebraic proof using square area}{}
		Consider a square of side length $(a + b)$, divided as shown below into four right-angled triangles and a small inner square.
		
		The area of the large square is:
		\[
		(a + b)^2
		\]
		
		The area can also be expressed as the sum of the areas of the four right-angled triangles and the small inner square:
		\[
		4 \cdot \left(\frac{1}{2}ab\right) + c^2
		\]
		
		Equating both expressions:
		\begin{align*}
			(a + b)^2 = 4 \cdot \left(\frac{1}{2}ab\right) + c^2\\
			a^2 + 2ab + b^2 = 2ab + c^2
		\end{align*}
		
		Subtract $2ab$ from both sides:
		\[
		a^2 + b^2 = c^2
		\]
		
		Which is exactly the Pythagorean theorem.
	\end{proof}
	
	\begin{remember}
		This proof shows that the theorem is not just a geometric rule, but also an algebraic identity based on area manipulation.
	\end{remember}
	
	{\small \printindex}
	
\end{document}


